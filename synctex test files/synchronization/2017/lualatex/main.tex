% lualatex hello-lualatex.tex
% see https://gist.github.com/bellbind/3316964
\documentclass{ltjsarticle}
\usepackage{luatexja-fontspec}
\usepackage{luacode}
\usepackage{hyperref}
\usepackage{listings}
\usepackage{moreverb}

\title{日本語UTF-8で\LaTeX : Lua\LaTeX}
\author{bellbind}
\date{2012年8月10日}

\begin{document}
\maketitle

{Lua\TeX}\footnote{\url{http://www.luatex.org/}}は、
{\TeX}拡張をプログラミング言語Lua\footnote{\url{http://www.lua.org/}}で
記述できるようにしたものです。
ディストリビューションtexlive 2012\footnote{\url{http://www.tug.org/texlive/}}
に標準で含まれており、
日本語文書でも利用可能です。

{Lua\LaTeX}での日本語文書用プリアンブルは以下のとおり。

\begin{lstlisting}[frame=single]
\documentclass{ltjsarticle}
\usepackage{luatexja-fontspec}
\end{lstlisting}

あとはtexファイルの文章をUTF-8で記述して、{\tt lualatex}コマンドで
コンパイルすれば直接PDFが出来上がります。

\begin{lstlisting}[frame=shadowbox]
lualatex foo.tex
\end{lstlisting}

ちなみにLua{\LaTeX}では、texソース中に、
Luaスクリプトを以下のように埋め込むことができます。

\begin{lstlisting}[frame=single]
\usepackage{luacode}

\luaexec{
tex.print("こんにちは{\\LaTeX}")
}
\end{lstlisting}

スクリプトコード中のバックスラッシュの扱いには注意が必要です。

\end{document}